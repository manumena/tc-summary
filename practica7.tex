\section{Práctica 7}

\subsection{}
\subsubsection{}
Uno solo

\subsubsection{}
Una sola

\subsubsection{}
Uno solo

\subsubsection{}
También uno. El servidor tiene un puerto abierto para atender a cada cliente que haga un pedido. Una vez atendido abre otro puerto y le comunica al cliente que a partir de ahora lo va a se comunican por ese otro puerto.

\subsection{}
\subsubsection{}
GET / HTTP 1.1 Host: dc.uba.ar

\subsubsection{}
HEAD /tdc HTTP 1.1 Host: dc.uba.ar

\subsubsection{}
GET IF-MODIFIED-SINCE *unafecha* /logo.jpg HTTP 1.1 Host: www.dc.uba.ar

\subsection{}
\subsubsection{}
HTTP 1.0
-> [syn]
<- [syn+ack]
-> [ack] + GET index HTTP 1.0
<- [fin] 200 OK  *index*
-> [fin+ack]
<- [ack]

En total lo de arriba son 3 RTT. Se repite lo de arriba para:
\begin{itemize}
\item style.css
\item searchline.png
\item home.png
\item search\_icon.gif
\end{itemize}
Luego, en total son 5*3 = 15 RTT

\subsubsection{}
$-> [syn]$
$<- [syn+ack]$
$-> [ack] + GET index HTTP 1.1$
$<- 200 OK  *index*$
$-> GET style.css HTTP 1.1$
$<- 200 OK  *style.css*$
$-> GET searchline.png HTTP 1.1$
$<- 200 OK  *searchline.png*$
$-> GET home.png HTTP 1.1$
$<- 200 OK  *home.png*$
$-> GET search_icon.gif HTTP 1.1$
$<- 200 OK  *search_icon.gif*$
$-> [fin]$
$<- [fin+ack]$
$-> [ack]$

En total son 7,5 RTT.
