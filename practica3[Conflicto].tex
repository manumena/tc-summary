\section{}

\subsection{}

\subsubsection{}
$t_{min} = 2 * Delay_{max} = 2 * 25.6 \mu s = 51.2 \mu s$

\subsubsection{}
$V_{tx} = 10 Mbps$

Regla de 3 simple. En un segundo transmito 10 Mb, en $51.2 \mu s$ transmito

$51.2 \mu s * 10Mbps = 51.2 \mu s * 10bp \mu s = 512bits = 64B$

\subsubsection{}
Se rellena con padding. Hay dos opciones para luego descartar el padding:

\begin{itemize}
\item En el header: en lugar de usar el campo type para multiplexar se usa como length tamaño y se usa LLC como multiplexador para la capa de red.
\item Se encarga la capa superior (por ejemplo IP length).
\end{itemize}

\subsubsection{}
Ambos sensan el medio en los respectivos momentos temporales e identifican que el medio está siendo utilizado. Ergo, esperan (1-persistente) hasta que el medio esté libre y luego transmiten. Ambos encontrarán el medio libre en momentos muy cercanos (la diferencia está dada por la distancia entre H2 y H3 con H1) y sus tramas colisionarán.

\subsubsection{}
H4 sensa el medio. Probablemente lo encuentre libre (por ejemplo si todavía no llega a sensar la información de la trama de H1 por estar a una distancia considerable), transmita y su trama colisione con la de H1.