\section{Práctica 6}

\subsection{}
Un puerto es una interfaz a través de la cual se pueden enviar y recibir los
diferentes tipos de datos.

Si no se implementaran los puertos, entonces las aplicaciones de dos hosts
distintos se comunicarían de manera broadcast, ya que al enviar un dato se
estaría enviando a todo quien escuche en ese host. Teniendo puertos es posible
definir de donde recibir y hacia donde enviar.

UDP es mas rápido ya que hace menos chequeos pero es menos seguro y no es
confiable. Es útil en caso de una aplicación de llamadas de voz o streaming.

TCP al contrario, tiene más delay, pero asegura que los datos enviados lleguen y
de manera segura, y en orden. Por lo que sirve para chats, o cualquier
aplicación donde es absolutamente necesario que los datos sean recibidos.

\setcounter{subsection}{4}
\subsection{}
\subsubsection{}
Hay 2 conexiones TCP: (20.232.1.1:80,1.2.3.4:5678) y (1.1.1.1:2222,3.3.3.3:4444)

\subsubsection{}
El criterio es fijarse en la dirección IP y puerto del destino y el orígen.

Para la conexión (20.232.1.1:80,1.2.3.4:5678) los segmentos: 1, 3, 4, 6, 8, 9,
11, 12 y 14

Para la conexión (1.1.1.1:2222,3.3.3.3:4444) los segmentos: 2, 5, 7, 10 y 13

\subsubsection{}
Puede verse que éste ocurre en el paquete 13, cuando el proceso en 3.3.3.3:4444
envía un RST. Esto se debe al paquete con flags SYN/ACK/URG recibido en la línea
10: recibir un SYN en una conexión sincronizada tiene como efecto el envío de un
RST, conforme a la especificación de TCP (para más detalles, referirse a la
página 71 del RFC 793). La consecuencia de este envío es que el receptor
procederá a liberar los recursos de la conexión de inmediato, sin ejercer el
algoritmo de cierre tradicional. A su vez, el host emisor del RST hace esto
mismo en reacción al paquete mal formado que recibió.

\subsubsection{}
\begin{itemize}
\item Del paquete 1 se desprende que 1.2.3.4:5678 está haciendo una apertura activa. Por ende, este paquete hace que su TCP se mueva de CLOSED a SYN-SENT.
\item El paquete 3 es una respuesta positiva del paquete 1 por parte del proceso 20.232.1.1:80. Para que esto ocurra, su TCP debe haber estado en LISTEN antes de poder contestar. Al recibir el SYN, la reacción consiste en trasladarse a SYN-RCVD y emitir un paquete SYN+ACK.
\item El primer host recibe luego este segmento y esto hace que se mueva a ESTABLISHED, enviando como respuesta el ACK final del three-way handshake (paquete 4).
\item Una vez procesado esto, el TCP de 20.232.1.1:80 se moverá también a ESTABLISHED. Luego tenemos que los paquetes 6, 8 y 9 corresponden a envío y reconocimiento de datos, por lo que no motivan transiciones entre estados.
\item En la porción final de la captura, vemos que se inicia el proceso de cierre de conexión. El segmento 11 nos muestra que 20.232.1.1:80 está haciendo un cierre activo, revirtiendo así su rol “pasivo” al momento de establecer la conexión. El efecto de este envío es transicionar al estado FIN-WAIT-1.
\item En la siguiente línea vemos que su interlocutor recibe este mensaje y reacciona respondien- do un paquete FIN+ACK que no sólo reconoce el primer FIN (puesto que su \#ACK es 10202) sino que además inicia el cierre de su stream de escritura. En consecuencia, su TCP colapsa dos estados, pasando de ESTABLISHED a LAST-ACK.
\item El segmento 14 reconoce correctamente el FIN de 1.2.3.4:5678. Al recibirlo, el TCP de dicho proceso transicionará a CLOSED, mientras que el de 20.232.1.1:80 se moverá a TIME-WAIT como respuesta al segmento 12. Asumiendo que el último ACK no se extravía en la red, allí permanecerá por 120 segundos (i.e., 2 MSLs) antes de pasar también a CLOSED.
\end{itemize}

\setcounter{subsection}{6}
\subsection{}
\subsubsection{}
\begin{itemize}
\item El host 2.71.82.81:823 pasa al estado FIN\_WAIT\_1
\item El host 3.14.15.92:654 recibe el FIN y envía su ACK, pasando al estado
CLOSE\_WAIT. El host 2.71.82.81:823 recibe el ACK por lo que pasa al estado
FIN\_WAIT\_2
\item 3.14.15.92:654 envía FIN pasando al estado LAST\_ACK
\item 2.71.82.81:823 envía el último ACK, pasando al estado TIME\_WAIT y
provocando que 3.14.15.92:654 pase al estado CLOSED. Luego de un tiempo de
espera, 2.71.82.81:823 también pasará a CLOSED.
\end{itemize}

\setcounter{subsection}{9}
\subsection{}
\setcounter{subsubsection}{1}
\subsubsection{}
Es luego del paquete 4, una vez recibido en SYN/ACK cuando el receptor puede
comenzar a enviar datos junto con un ACK que finaliza el three-way handshake

\subsection{}
\subsubsection{}
\begin{enumerate}
\item 192.168.0.100 pasa de CLOSED a SYN\_SENT
\item 157.92.27.21 pasa de LISTEN a SYN\_RCVD
\item 192.168.0.100 pasa a ESTABLISHED al enviar el ACK y 157.92.27.21 también
lo hace al recibirlo
\item Para el resto la conexión se mantiene establecida
\end{enumerate}

\subsection{}
\subsubsection{}
\begin{enumerate}
\item 157.92.27.21:80 - 10.80.1.10:35336 	SYN,ACK
\item 10.80.1.10:35336 - 157.92.27.21:80 	ACK
\item 157.92.27.21:80 - 10.80.1.10:35336 	FIN
\item 10.80.1.10:8080 - 200.11.54.101:32154 SYN
\item 200.11.54.101:32154 - 10.80.1.10:8080 SYN,ACK
\item 10.80.1.10:8080 - 200.11.54.101:32154 ACK
\end{enumerate}

\subsubsection{}
$t_1 \geq t_0 + 2 $ segment lifetimes
