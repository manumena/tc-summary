\section{}

\setcounter{subsection}{1}
\subsection{}
\subsubsection{}
Dependiendo de la version el formato del header es distinto, por lo que no saber que version es puede provocar no poder leer correctamente los campos.

\setcounter{subsubsection}{2}
\subsubsection{}
El tamaño máximo de un paquete IP es 65535 bytes. El campo lenght define este tamaño.

\subsection{}
\subsubsection{}
Tiene 3 interfaces: FastEthernet, Wireless0 Connection y Wirless1 Connection

\subsubsection{}
\begin{itemize}
\item Physical Address es la dirección MAC
\item IP Address es la direccion IP
\item Subnet Mask es la máscara de subred que se aplica
\item Default Gateway es el router por defecto
\end{itemize}

\subsection{}
\subsubsection{}
Es solo pasar los /N a XXX.XXX.XXX.XXX

/22 = 255.255.252.0

/23 = 255.255.254.0

/24 = 255.255.255.0

/25 = 255.255.255.127

\subsubsection{}
135.46.56.0/22 = 135.46. 0011 1000 .0 / 255.255.252.1111 1100 = 135.46.56.0

Capacidad máxima de hosts esta dada por /22, son todos las direcciones que podemos meter sobre los 10 digitos que la máscara anula. Osea $2^{10} = 1024$ hosts.

\subsubsection{}
\begin{tabular}{|c|c|c|}
\hline
& A & B \\ \hline
135.46.57.14 & Interface1 & descarta \\ \hline
135.46.63.10 & Interface0 & descarta \\ \hline
135.46.52.2 & 135.46.62.100 & descarta \\ \hline
208.70.188.15 & 135.46.62.100 & descarta \\ \hline
135.46.62.62 & Interface0 & descarta \\ \hline 
192.53.40.7 & 135.46.60.100 & Interface1 \\ \hline 
192.53.56.7 & 135.46.62.100 & descarta \\ \hline 
\end{tabular}

\subsection{}
PC1: 172.16.5.

\subsection{}
\subsubsection{}
Pueden direccionarse 256 - 2 hosts.

\subsubsection{}
2 subredes: 128 - 2 hosts.

4 subredes: 64 - 2 hosts.

8 subredes: 32 - 2 hosts.

\subsection{}
\subsubsection{}
Red A: 172.16.5.0/27 (32)

Red B: 172.16.5.32/29 (8) Dejo 8 porque se necesitan 1 host + 2 routers + 2 de broadcast y localhost

Red C: No puedo usar 15.16.5.40/27 (128) porque se solaparía con 17.16.5.64/22 que ya existe

